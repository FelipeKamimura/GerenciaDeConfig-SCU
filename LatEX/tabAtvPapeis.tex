% Table generated by Excel2LaTeX from sheet 'Planilha2'
    \begin{longtable}[c]{|p{1.89em}|p{13.555em}|p{17.61em}|p{6.22em}|}
    %%%
   %%%
  %%% firsthead -- essa seção aparece apenas no primeiro cabeçalho.
  %%%
  \caption{Enumeração e descrição das atividades vinculados com os papeis\label{tabAtvPapeis}} \\
  \hline
  {} & {ATIVIDADE} & {DESCRIÇÃO DE ATIVIDADES} & {PAPÉIS} \\
  \hline\hline
  \endfirsthead
  %%%
  %%% head -- essa seção aparece nos demais cabeçalhos.
  %%%
  \caption[]{Enumeração e descrição das atividades vinculados com os papeis (continuação)} \\
  \hline
  {} & {ATIVIDADE} & {DESCRIÇÃO DE ATIVIDADES} & {PAPÉIS} \\
  \hline\hline
  \endhead
 
    \textbf{1.1} & Reunião de iniciação & Reunião para marcar os encontros com os stakeholders para que todos saibam o que é esperado dessa primeira fase, serão definidos os tempos para a execução das sprints e como podem ser realizadas & Stakeholders \\
    \midrule
    \textbf{1.2} & Definição da viabilidade do projeto & Verificar que com o orçamento proposto e tecnologias prontas para serem utilizadas são suficientes para a execução do projeto & Stakeholders \\
    \midrule
    \textbf{1.3} & Definição de escopo geral & Definir linguagem, limites e restrições, que problemas que devem ser resolvidos e como serão resolvidos & Stakeholders \\
    \midrule
    \textbf{1.4} & Reunião com o Product Owner & Reunião para verificar se o cliente está de acordo com o escopo definido & Stakeholders \\
    \midrule
    \textbf{1.5} & Análise de requisitos & Onde serão realizados os modelos de casos de uso para auxiliar na análise dos requisitos necessários e verificar as atividades e funcionalidades que precisam estar presentes no software a ser desenvolvido & Scrum Team \\
    \midrule
    \textbf{1.6} & Definição dos atores & A partir dos casos de usos são listados e verificados os atores envolvidos no sistema & Scrum Team \\
    \midrule
    \textbf{1.7} & Definição da arquitetura & São especificados os casos de uso gerais sobre as funcionalidades essenciais do sistema & Scrum Team \\
    \midrule
    \textbf{1.8} & Especificação dos requisitos & Com as funcionalidades definidas os requisitos funcionais e não funcionais podem ser mais especificados com o fim de validar a necessidade, se não estão ambíguos, e se suprem as demandas & Scrum Team \\
    \midrule
    \textbf{1.9} & Avaliar e validar o documento de requisitos & A partir dos relatórios dos requisitos, os requisitos são analisados em busca de verificar se possuem rastreabilidade, se estão íntegros, correlacionados e se o documento de requisitos são & Stakeholders \\
    \midrule
    \textbf{1.10} & Prototipação de interface & Com o uso de um design sprint será feito o estudo de um planejamento sobre a interface gráfica do sistema, para certificar que todos os requisitos estarão sendo preenchidos & Scrum Team \\
    \midrule
    \textbf{1.11} & Cálculo de pontos por função & Com o uso dos caso de uso, será realizado um cronograma inicial para se obter uma estimativa de prazo de execução do projeto & Scrum Team \\
    \midrule
    \textbf{1.12} & Reunião de revisão (transição) & Reunião de revisão de fase onde é verificado se o planejado foi executado, se há necessidade de mudança na execuções das sprints antes do inicio da próxima fase. É também elaborado o product backlog a partir de tudo que foi elicitado e também definido suas prioridades & Stakeholders \\
    \midrule
    \textbf{2.1} & Reunião de iniciação & Definido o sprint backlog para que os parâmetros das sprints sejam passados para o scrum team & Stakeholders \\
    \midrule
    \textbf{2.2} & Definição da arquitetura de hardware & São definidos todos os componentes (entregas) referentes ao projeto & Scrum Team \\
    \midrule
    \textbf{2.3} & Definição da arquitetura de software & Através do diagrama de pacote são definidos como os pacotes serão estruturados dentro da arquitetura do sistema & Scrum Team \\
    \midrule
    \textbf{2.4} & Definição do modelo de banco de dados & Nessa atividade é estudado os itens que precisam estar no banco de dados, se será relacional ou não relacional, atributos, entidades, chaves e relacionamentos & Scrum Team \\
    \midrule
    \textbf{2.5} & Definição do diagrama de classes & Desenvolvimento dos diagramas de classes para que se tenha a visão geral e como as classes serão relacionadas umas com as outras, seus métodos e organização em relação aos objetos, além de separar elementos de design dos elementos de sistema & Scrum Team \\
    \midrule
    \textbf{2.6} & Definição dos diagramas de atividades & Definir o fluxo do funcionamento e controle das atividades realizadas pelo sistema e como se interligam & Scrum Team \\
    \midrule
    \textbf{2.7} & Definição do diagrama de implantação & Desenvolvimento do Diagrama de Implantação que especifica os requisitos mínimos para o funcionamento do sistema tais como as bibliotecas e softwares necessários, assim como as definições de hardware & Scrum Team \\
    \midrule
    \textbf{2.8} & Modelagem da interface com o usuário & Diagramação para planejamento da tela e disposição dos itens na interface e estudo de como o usuário pode ter uma boa experiencia de utilização. Usa como base o protótipo inicial porém na aplicação de heurísticas e experiencia de usuário' & Scrum Team \\
    \midrule
    \textbf{2.9} & Desenvolvimento e codificação & Codificação & Scrum Team \\
    \midrule
    \textbf{2.10} & Descrição do código & São documentadas as formas e padrões adotados na codificação, além de descrever determinados blocos de código-fonte & Scrum Team \\
    \midrule
    \textbf{2.11} & \multicolumn{1}{r|}{} & Reunião de revisão de fase onde é verificado se o planejado foi executado, se há necessidade de mudança na execuções das sprints antes do inicio da proxima fase. Se o software atende ao escopo especificado & Stakeholders \\
    \midrule
    \textbf{3.1} & \multicolumn{1}{r|}{} & Reunião de inicio de fase, onde será definido como serão realizados os testes & Stakeholders \\
    \midrule
    \textbf{3.2} & Ciclo de testes & \multirow{1}{r|} & Realização dos testes e correção no software, se necessário. Os testes devem ser relatados & Scrum Team \\
\cmidrule{1-2}\cmidrule{4-4}    \textbf{3.3} & Teste de Componentes & \multicolumn{1}{r|}{} & Scrum Team \\
\cmidrule{1-2}\cmidrule{4-4}    \textbf{3.4} & Teste de Sistema & \multicolumn{1}{r|}{} & Scrum Team \\
\cmidrule{1-2}\cmidrule{4-4}    \textbf{3.5} & Teste de Aceitação & \multicolumn{1}{r|}{} & Scrum Team \\
    \midrule
    \textbf{3.6} & Reunião de revisão (transição) & Reunião de finalização do software, realizada com os os stakeholders a fim de validar o sistema de acordo com o que era esperado & Stakeholders \\
    \midrule
    \textbf{4.1} & Reunião de iniciação & Reunião de inicio de fase, onde é passado aos stakeholders como serão os procedimentos finais de desenvolvimento e como deve ser documentado as atualizações do sistema & Stakeholders \\
    \midrule
    \textbf{4.2} & Implementação dos Manuais & Desenvolvimento dos Manuais do usuário e sistema para treinamento ou ajuda adicional para usuários, além do desenvolvimento dos Termos de Uso para o software & Scrum Team \\
    \midrule
    \textbf{4.3} & Instalação do sistema & Instalação do sistema no ambiente requisitado pelo Product Owner seguindo as especificações & Scrum Team \\
    \midrule
    \textbf{4.4} & Migração dos dados & Transferência de dados antigos para o banco de dados arquitetado resumindo no mesmo atualizado & Scrum Team \\
    \midrule
    \textbf{4.5} & Treinamentos dos usuários & Relatório de resultados do treinamento dos usuários que foram qualifiados para o uso do sistema & Scrum Team \\
    \midrule
    \textbf{4.6} & Reunião de revisão (finalização) & Documentar considerações levantadas na reunião sobre o desenvolvimento das sprints e se o produto final está conforme o TAP, se atende o escopo e se o cliente final necessita de mais alguma alteração & Stakeholders \\
    \midrule
    \textbf{5.1} & Reunião de iniciação & Reunião de inicio da fase de manutenção e melhorias no sistema, definindo como o cronograma de rotinas será feito e especificação de relatórios de melhorias e bugs encontrados & Stakeholders \\
    \midrule
    \textbf{5.2} & Definição de Rotinas de manutenção e testes & Definição do cronograma e finalidades de manutenção e testes do sistema & Scrum Master \\
    \midrule
    \textbf{5.3} & Definição dos bugs do sistema & Relatório geral com o status de cada bug (se foi ou não resolvido, como foi resolvido, descrição do bug) & Scrum Team \\
    \midrule
    \textbf{5.4} & Definição de resolução dos bugs & Relatório geral das resoluções dos bugs encontrados no sistema\newline{} & Scrum Team \\
    \midrule
    \textbf{5.5} & Controle de pedidos de alterações & Status de cada pedido, se a melhoria sera ou não desenvolvida & Scrum Master \\
    \midrule
    \textbf{5.6} & Rastreamento do progresso & Relatório geral com o status de cada melhoria desenvolvida e seus respectivos feedback & Scrum Team \\
    \midrule
    \textbf{5.7} & Medir os resultados das atividades & observações: serve para verificar se as atividades alcançaram realmente o objetivo delas & Scrum Team \\
    \midrule
    \textbf{5.8} & Controle das versões do sistema & Desenvolvimento do documento das versões com os Relatórios de melhorias e resoluções de bugs no sistema, obtendo um controle das versões do sistema & Scrum Master \\
    \midrule
    \textbf{5.9} & Reunião de revisão (transição) & Documentar considerações levantadas na reunião sobre o desenvolvimento das sprints & Stakeholders \\
    \bottomrule
    \end{longtable}
